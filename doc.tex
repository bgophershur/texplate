% Hebrew document using polyglossia
\documentclass{article}

% Compiler must be XeLaTeX or LuaLaTeX
\usepackage{fontspec}
\usepackage{polyglossia}
\usepackage{enumitem}
\setmainlanguage{hebrew}
\setotherlanguage{english}

\setlist[itemize]{label=-}

% Use a Hebrew-supporting font
\newfontfamily\hebrewfont[Script=Hebrew]{David CLM}

% Layout and math
\usepackage[top=2cm,bottom=2cm,left=2.5cm,right=2cm]{geometry}
\usepackage{siunitx}
\sisetup{
    separate-uncertainty = true,
    per-mode = symbol
}
\usepackage{graphicx}
\graphicspath{{graphics/}}
\usepackage{float}
\usepackage{amsmath}
\usepackage{xfp}

% University info
\newcommand{\universityName}{Name of University}
\newcommand{\facultyName}{Name of Faculty}
\newcommand{\departmentName}{Name of Department}

% Experiment info
\newcommand{\experimentTitle}{Title of Experiment}
\newcommand{\experimentLabel}{Experiment Label}

% Author + Instructor
\newcommand{\studentName}{Your Name Here}
\newcommand{\instructorName}{Instructor Name}
\newcommand{\labInstructorLabel}{Lab Instructor Label}

% Date (you can override \today if needed)
\newcommand{\reportDate}{\today}
% Flipping equation numbering parentheses
\renewcommand{\theequation}{\thesection.\arabic{equation}}

% Hebrew-safe image command
\newcommand{\hebrewincludegraphics}[2][]{%
  \begin{LTR}
    \includegraphics[#1]{#2}%
  \end{LTR}%
}

\begin{document}
  \begin{titlepage}
    \centering
    {\LARGE \universityName\par}
    \vspace{1em}
    {\Large \facultyName\par}
    \vspace{1em}
    {\large \departmentName\par}
    \vspace{4em}
    {\Huge \textbf{\experimentLabel:}\par}
    \vspace{1em}
    {\huge \experimentTitle\par}
    \vspace{6em}
    {\Large \textbf{מאת:}\par}
    \vspace{0.5em}
    {\large \studentName\par}
    \vspace{3em}
    {\Large \textbf{\labInstructorLabel:}\par}
    \vspace{0.5em}
    {\large \instructorName\par}
    \vfill
    {\Large תאריך: \reportDate\par}
\end{titlepage}

  \tableofcontents
  \newpage
  \input{sections/abstract.tex}
  \input{sections/introduction.tex}
  \input{sections/theory.tex}
  \input{sections/method.tex}
  \input{sections/results.tex}
  \input{sections/analysis.tex}
  \input{sections/conclusion.tex}
  \appendix
  \section*{עבודת הכנה}
\end{document}